\documentclass[12pt]{article}

\usepackage{sbc-template}

\usepackage{graphicx,url}

\usepackage[brazil]{babel}   
%\usepackage[latin1]{inputenc}  
\usepackage[utf8]{inputenc}  
% UTF-8 encoding is recommended by ShareLaTex

\usepackage{mdframed}
\usepackage{minted}

\usepackage{amssymb}
     
\sloppy

\title{Trabalho Prático 3 - Tradução de código intermediário do front-end de SmallL para TAM}

\author{Hugo Araujo de Sousa\\(2013007463)}

\address{
  Compiladores I (2017/2) \\
  Departamento de Ciência da Computação \\
  Universidade Federal de Minas Gerais (UFMG)
  \email{hugosousa@dcc.ufmg.br}
}

\begin{document} 

\maketitle
     
\begin{resumo}
  Este trabalho tem como objetivo o desenvolvimento de um tradutor da representação
  intermediária da linguagem SmallL, como definido no Trabalho Prático II da disciplina
  Compiladores I, para a máquina virtual TAM.
\end{resumo}

\section{INTRODUÇÃO}

No Trabalho Prático II (TPII), foi desenvolvido um front-end completo para a linguagem SmallL. Dessa forma, todas as fases
de análise léxica, sintática, semântica e geração de código intermediário foram completadas. O próximo passo, então,
no processo de compilação de um programa escrito em SmallL é o de tradução do código intermediário gerado pelo front-end
criado no TPII para alguma máquina alvo.

Como a etapa de tradução é dependente da máquina alvo, o primeiro passo é determinar para qual máquina o programa em
representação intermediária deverá ser traduzido. Para esse trabalho, a máquina escolhida foi a máquina virtual TAM. Para
informações sobre a máquina TAM, seu conjunto de instruções e registradores, ver \cite{Watt:2007:PLP:1557477}.

\section{METODOLOGIA} \label{sec:met}

A primeira etapa no desenvolvimento do trabalho foi a de realizar uma pequena alteração do código do TPII, uma vez que o
código intermediário gerado por esse não possui qualquer anotação sobre os tipos das variáveis. Sem essa informação,
não é possível alocar espaço em memória para cada variável.

Logo, o TPII modificado imprime no cabeçalho do código intermediário gerado anotações de tipo das variáveis e o tradutor
realiza a alocação de memória para as mesmas.

Em seguida, foi necessário identificar todos os tipos de quádruplas presentes no
código intermediário. Esses tipos são apresentados a seguir:

\begin{enumerate}
 \item \textbf{Saltos condicionais:} Desvios de código baseados em uma condição booleana. São representados por
 instruções "if" e "iffalse".
 
 \item \textbf{Saltos incondicionais:} Desvios de código independentes de qualquer condição booleana. São representados
 por instruções do tipo "goto".
 
 \item \textbf{Indexação de array como l-value:} Uma variável ou valor é copiado para um elemento de array.
 
 \item \textbf{Indexação de array como r-value:} Um elemento de array é copiado para uma variável.
 
 \item \textbf{Atribuição simples:} Valor número ou de uma variável é atribuído a outra variável.
 
 \item \textbf{Atribuição aritmética:} Resultado de uma expressão aritmética é atribuída a uma variável.
 
 \item \textbf{Atribuição unária:} Resultado de uma operação unária é atribuída a uma variável.
\end{enumerate}

Com posse desses tipos, o processo de tradução se reduz a identificar quais são as instruções TAM necessárias 
para executar cada uma das quádruplas presentes no código intermediário. A metodologia usada usou como referência
\cite{Aho:86}.

De maneira geral, a estrutura do algoritmo é a de ler todas as quádruplas, efetuar a tradução de cada uma e, por fim, 
efetuar o processo de Backpatching para obter os endereços das labels de desvio ainda não definidas.

\section{CÓDIGO FONTE}

O código fonte pode ser encontrado na pasta \textbf{src} no diretório do projeto no
GitHub\footnote{https://github.com/ha2398/compiladores1-tps/tree/master/tp3}. A pasta \textbf{tp2-modified} contém
o código fonte da versão modificada do TPII, como dito na Seção \ref{sec:met}.

O tradutor foi implementado utilizando a linguagem Python 3 e o código foi separado em dois arquivos:

\begin{enumerate}
 \item \textbf{quadruple.py:} Define a classe \textbf{Quadruple} que representa uma quádrupla.
 
 \item \textbf{translator.py:} Implementa o tradutor propriamente dito.
\end{enumerate}

\section{EXECUÇÃO}

Para executar o tradutor, primeiramente é necessário fornecer como entrada um arquivo em código intermediário de SmallL
gerado através da versão modificada do TPII. Da raiz do diretório do projeto, fazemos:

\begin{center}
  cd tp2-modified/src \\
  make \\
  java main.Main $<$ arquivo\_entrada
\end{center}

Onde \textbf{arquivo\_entrada} é o arquivo em SmallL para o qual deseja-se gerar código intermediário. Após a execução dos 
comandos, será gerado um arquivo \textbf{code.out} contendo as quádruplas correspondentes ao código de entrada.

Com o código intermediário em mãos, podemos executar o tradutor. Para isso, da raiz do diretório do projeto, fazemos:

\begin{center}
 cd src \\
 ./translator.py arquivo\_entrada arquivo\_saida
\end{center}

Onde \textbf{arquivo\_entrada} é o arquivo em código intermediário gerado no passo anterior e \textbf{arquivo\_saida}
é o arquivo onde o resultado da tradução, isto é, o código TAM correspondente ao código intermediário de entrada
deve ser salvo. O tradutor também lista o arquivo de entrada na saída padrão.

\section{TESTES E RESULTADOS}

Para ilustrar o funcionamento do tradutor, são apresentados dois exemplos de tradução, onde são mostrados primeiramente
o código intermediário de entrada e o código TAM de saída.

\subsection{code1.out}

\begin{itemize}
 \item Código intermediário:\\
 
  \begin{mdframed}[linecolor=black, leftline=false, rightline=false]
    \inputminted[linenos, fontsize=\footnotesize]{text}{../input/code1.out}
  \end{mdframed}
  
  \mbox{}
 
 \item Código TAM: \\
 
 \begin{mdframed}[linecolor=black, leftline=false, rightline=false]
    \inputminted[linenos, fontsize=\footnotesize]{text}{../output/code1.tam}
  \end{mdframed}
\end{itemize}

\subsection{code3.out}

\begin{itemize}
 \item Código intermediário: \\
 
  \begin{mdframed}[linecolor=black, leftline=false, rightline=false]
    \inputminted[linenos, fontsize=\footnotesize]{text}{../input/code3.out}
  \end{mdframed}
 
 \mbox{}
 
 \item Código TAM: \\
 
 \begin{mdframed}[linecolor=black, leftline=false, rightline=false]
    \inputminted[linenos, fontsize=\footnotesize]{text}{../output/code3.tam}
  \end{mdframed}
\end{itemize}


\section{CONCLUSÃO}

O trabalho em questão faz paralelo com um trabalho realizado para a disciplina Software Básico, onde um tradutor
também deveria ser implementado. Dessa forma, uma certa familiaridade com o tema já existia. Entretanto, para esse
trabalho, o tópico foi mais aprofundado, levantando muitas questões que antes não foram abordadas, como organização
de memória, verificação de tipos, etc.

É importante ressaltar que a maior dificuldade foi a de estudar a fundo a máquina alvo TAM. Não havia nenhum conhecimento
sobre a mesma e há pouca documentação online.

Alguns pontos levantaram muita dúvida durante a implementação, por exemplo, no manual da máquina vemos que somente 
há suporte para operações aritméticas com valores inteiros, o que é uma grande falha na especificação, uma vez que a SmallL
permite o uso de variáveis de ponto flutuante. 

De maneira geral, o trabalho pode ser concluído com sucesso e os testes se mostraram correspondentes ao que era desejado.

\section{REFERÊNCIAS}

\bibliographystyle{sbc}
\bibliography{sbc-template}

\end{document}
