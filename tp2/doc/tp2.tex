 % % % % % % %% %PREAMBULO% % % % % % % % % % %
\documentclass[a4paper,12pt]{report}

\usepackage{fancyhdr}
\usepackage[utf8]{inputenc}
\usepackage{minted}
\usepackage{graphicx}
\usepackage[portuguese]{babel}
\usepackage{mdframed}

\usepackage{geometry}
 \geometry{
 a4paper,
 total={170mm,257mm},
 left=20mm,
 top=20mm,
 }
 
\usemintedstyle{bw}

\pagestyle{fancy}\lhead{ Um Front-End Completo do Compilador de SmallL } \rhead{DCC053 - Trabalho Prático 2}
\renewcommand{\contentsname}{Índice}
\renewcommand\thesection{\arabic{section}}

 % % % % % % %% %DOCUMENTO% % % % % % % % % % %

\begin{document}

% Capa
\begin{titlepage}
	\centering
	{\scshape\LARGE Universidade Federal de Minas Gerais \par}
	{\scshape\Large Instituto de Ciências Exatas \par}
	{\scshape\Large Departamento de Ciência da Computação \par}
	\vfill
	{\huge\bfseries \textbf{Compiladores I - DCC053} \par}
	{\huge\bfseries \textbf{Trabalho Prático 2\\Um Front-End Completo do Compilador de SmallL} \par}
	\vfill
	{\Large Hugo Araújo de Sousa\\(2013007463)\par}
	\vfill
	{\large 2º Semestre de 2017\par}
\end{titlepage}

\newpage
\tableofcontents
\newpage

\section{Descrição do Problema}

Em projetos de construção de compiladores, dizemos que o \textbf{front-end} do compilador é
a parte do mesmo responsável pela análise do código fonte. Sendo assim, essa é a parte dependente
da linguagem fonte. Ela consiste geralmente de três etapas:

\begin{enumerate}
 \item Análise Léxica: Lê o o programa fonte agrupando os símbolos que o compõem em sequências significativas.
 Para cada uma dessas sequências, armazena-se informações importantes para etapas posteriores.
 
 \item Análise Sintática: Utiliza a saída da análise léxica para impor uma estrutura gramatical ao programa
 fonte. 
 
 \item Análise Semântica: Verifica a consistência semântica do programa fonte, estabelecida na própria definição
 da linguagem. Nessa etapa ocorre a verificação de tipos.
 
 \item Geração de Código Intermediário: Gera uma representação intermediária de baixo nível ou do tipo linguagem
 de máquina do programa fonte. Essa representação deve ser facilmente traduzida para a máquina alvo.
\end{enumerate}

Nesse trabalho, é construído um front-end para a mini-linguagem SmallL, definida pela gramática mostrada a seguir.

\begin{mdframed}[linecolor=black, leftline=false, rightline=false]
  \inputminted[linenos, fontsize=\footnotesize]{text}{grammar.txt}
\end{mdframed}

\begin{center}
 Gramática da linguagem SmallL. 'e' indica a string vazia.
\end{center}

\section{Metodologia}

O front-end segue a implementação do Livro Texto da disciplina. Dessa forma, o código (escrito na linguagem
Java), foi dividido em cinco pacotes: \textbf{main}, \textbf{lexer}, \textbf{symbols}, \textbf{parser} e
\textbf{inter}. Esses pacotes são descritos na Seção \ref{sec:cf}. Cada um deles representa uma etapa do front-end.

\subsection{Análise Léxica}

A análise léxica é feita de forma tradicional, lendo os caracteres do programa de entrada e tentando agrupá-los 
em tokens reconhecidos pela gramática da linguagem. Para armazenar informações de símbolos, a tabela de símbolos
foi implementada utilizando-se uma tabela Hash.

O analisador léxico armazena o número da linha do programa fonte em que se encontra a fim de fornecer essa
informação ao usuário no caso de erro.

\subsection{Análise Sintática}

O analisador sintático atua sempre em comunicação com o analisador léxico, requisitando tokens e usando-os
para construir a estrutura gramatical definida pela gramática da linguagem. Para esse trabalho a análise sintática
foi implementada utilizando o algoritmo de análise descendente. 

\subsection{Análise Semântica}

A análise semântica implementada no trabalho se resume à verificação de tipos, que é então feita de forma 
integrada com a geração de código intermediário.

\subsection{Geração de Código Intermediário}

A geração de código intermediário segue o esquema de geração de código com três endereços. Esse código 
suporta expressões ariméticas, expressões de desvio e declarações.

\section{Código Fonte} \label{sec:cf}

Os pacotes do projeto são descritos a seguir. Todo o código fonte pode ser obtido em um repositório do GitHub
\footnote{https://github.com/ha2398/compiladores1-tps/tree/master/tp2}.

\begin{itemize}
 \item \textbf{main} Programa principal, realiza a integração de todos os outros módulos.
 
 \item \textbf{lexer} Implementa a análise léxica do front-end.
 
 \item \textbf{parser} Analisador semântico.
 
 \item \textbf{symbols} Análise semântica, notadamente verificação de tipos.
 
 \item \textbf{inter} Geração de código intermediário de três endereços.
\end{itemize}

\section{Compilação e Uso}

A fim de facilitar a compilação do projeto, um arquivo Makefile é fornecido na diretório src. Para
compilar o projeto, basta estar no diretório src e executar o comando \textbf{make}.

Já para executar o front-end, executa-se, na pasta src, o seguinte comando:

\begin{center}
 java main.Main $ < $ arquivo\_entrada
\end{center}

Onde arquivo\_entrada indica o nome do arquivo fonte a ser analisado.

\section{Testes e Resultados}

A fim de verificar o funcionamento correto do projeto, alguns testes foram desenvolvidos. Suas entradas 
e saídas são apresentadas a seguir.

\subsection{right1.txt}

Descreve um arquivo de teste simples, de sintaxe correta, que simplesmente executa algumas declarações e 
comandos de loop, atribuições e expressões aritméticas.

\begin{itemize}
 \item Entrada
 
 \begin{mdframed}[linecolor=black, leftline=false, rightline=false]
    \inputminted[linenos, fontsize=\footnotesize]{text}{../input/right1.txt}
\end{mdframed}
 
 \item Saída

 \begin{mdframed}[linecolor=black, leftline=false, rightline=false]
    \inputminted[linenos, fontsize=\footnotesize]{text}{../output/right1.txt}
  \end{mdframed}
  
\end{itemize}

\subsection{right2.txt}

Implementação do algoritmo Insertion Sort na linguagem SmallL. A sintaxe está correta.

\begin{itemize}
 \item Entrada
 
 \begin{mdframed}[linecolor=black, leftline=false, rightline=false]
    \inputminted[linenos, fontsize=\footnotesize]{text}{../input/right2.txt}
\end{mdframed}
 
 \item Saída

 \begin{mdframed}[linecolor=black, leftline=false, rightline=false]
    \inputminted[linenos, fontsize=\footnotesize]{text}{../output/right2.txt}
  \end{mdframed}
  
\end{itemize}

\subsection{right3.txt}

Calcula o décimo número da sequência de Fibonacci. A sintaxe está correta.

\begin{itemize}
 \item Entrada
 
 \begin{mdframed}[linecolor=black, leftline=false, rightline=false]
    \inputminted[linenos, fontsize=\footnotesize]{text}{../input/right3.txt}
\end{mdframed}
 
 \item Saída

 \begin{mdframed}[linecolor=black, leftline=false, rightline=false]
    \inputminted[linenos, fontsize=\footnotesize]{text}{../output/right3.txt}
  \end{mdframed}
  
\end{itemize}

Os arquivos de teste a seguir tratam-se dos mesmos arquivos mostrados acima, porém com pequenos erros de sintaxe.

\subsection{wrong1.txt}

\begin{itemize}
 \item Entrada
 
 \begin{mdframed}[linecolor=black, leftline=false, rightline=false]
    \inputminted[linenos, fontsize=\footnotesize]{text}{../input/wrong1.txt}
\end{mdframed}
 
 \item Saída

 \begin{mdframed}[linecolor=black, leftline=false, rightline=false]
    \inputminted[linenos, fontsize=\footnotesize]{text}{../output/wrong1.txt}
  \end{mdframed}
  
\end{itemize}

\subsection{wrong2.txt}

\begin{itemize}
 \item Entrada
 
 \begin{mdframed}[linecolor=black, leftline=false, rightline=false]
    \inputminted[linenos, fontsize=\footnotesize]{text}{../input/wrong2.txt}
\end{mdframed}
 
 \item Saída

 \begin{mdframed}[linecolor=black, leftline=false, rightline=false]
    \inputminted[linenos, fontsize=\footnotesize]{text}{../output/wrong2.txt}
  \end{mdframed}
  
\end{itemize}

\subsection{wrong3.txt}

\begin{itemize}
 \item Entrada
 
 \begin{mdframed}[linecolor=black, leftline=false, rightline=false]
    \inputminted[linenos, fontsize=\footnotesize]{text}{../input/wrong3.txt}
\end{mdframed}
 
 \item Saída

 \begin{mdframed}[linecolor=black, leftline=false, rightline=false]
    \inputminted[linenos, fontsize=\footnotesize]{text}{../output/wrong3.txt}
  \end{mdframed}
  
\end{itemize}

\subsection{wrong4.txt}

\begin{itemize}
 \item Entrada
 
 \begin{mdframed}[linecolor=black, leftline=false, rightline=false]
    \inputminted[linenos, fontsize=\footnotesize]{text}{../input/wrong4.txt}
\end{mdframed}
 
 \item Saída

 \begin{mdframed}[linecolor=black, leftline=false, rightline=false]
    \inputminted[linenos, fontsize=\footnotesize]{text}{../output/wrong4.txt}
  \end{mdframed}
  
\end{itemize}

\section{Conclusão}

O trabalho foi desenvolvido sem dificuldades a partir do código já fornecido na especificação. Apenas algumas
correções de sintaxe foram necessárias juntamente com ajustes para listagem de código fonte. De maneira geral,
o trabalho em questão permitiu um maior entendimento sobre o front-end de um compilador, fornecendo uma visão
mais detalhada das etapas que o compõem.

\section{Referências}

\begin{itemize}
 \item Aho, A.V.; Sethi, R.; Ullman, J.D. Compilers Principles, Techniques, and
 Tools, Addison Wesley, 1986.
\end{itemize}


\end{document}
