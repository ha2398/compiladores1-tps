\documentclass[12pt]{article}

\usepackage{sbc-template}

\usepackage{graphicx,url}

\usepackage[brazil]{babel}   
%\usepackage[latin1]{inputenc}  
\usepackage[utf8]{inputenc}  
% UTF-8 encoding is recommended by ShareLaTex

\usepackage{mdframed}
\usepackage{minted}

\usepackage{amssymb}
     
\sloppy

\title{Trabalho Prático 4 - Compilador Integrado SmallL para TAM}

\author{Hugo Araujo de Sousa\\(2013007463)}

\address{
  Compiladores I (2017/2) \\
  Departamento de Ciência da Computação \\
  Universidade Federal de Minas Gerais (UFMG)
  \email{hugosousa@dcc.ufmg.br}
}

\begin{document} 

\maketitle
     
\begin{resumo}
  Nesse trabalho é desenvolvido um compilador integrado completo para a linguagem SmallL,
  tendo a máquina TAM como alvo.
\end{resumo}

\section{INTRODUÇÃO}

Através dos Trabalhos Práticos II e III (TPII e TPIII) da disciplina Compiladores I foi possível desenvolver um front-end e back-end,
respectivamente, da linguagem de programação SmallL tendo como máquina alvo a máquina virtual TAM \cite{Watt:2007:PLP:1557477}.
Nesse trabalho, a tarefa é a de integrar essas duas partes para desenvolver um compilador completo.

\section{METODOLOGIA} \label{sec:met}

Tendo os Trabalhos Práticos II e III em mãos, para criar o compilador integrado foi necessário somente escrever um programa
que chama cada um dos programas criados nos trabalhos práticos anteriores e utiliza a saída de um como entrada do outro.

Dessa forma, o arquivo fonte escrito em SmallL é passado como entrada para o programa do TPII e o arquivo em código intermediário
de saída do TPII é passado como entrada para o programa do TPIII. A saída do TPIII é então o programa SmallL inicial escrito
na linguagem de máquina TAM.

\section{CÓDIGO FONTE}

Todo o código do Trabalhos Práticos II e III foi agrupado na pasta \textbf{src}, nos diretórios \textbf{front-end} e
\textbf{back-end}, respectivamente. Além disso, o compilador integrado é implementado através de um script escrito em Python 3,
no arquivo \textbf{compiler.py}.

O código fonte desse trabalho pode ser encontrado no diretório do projeto no
GitHub\footnote{https://github.com/ha2398/compiladores1-tps/tree/master/tp4}.

\section{EXECUÇÃO}

Para executar o compilador, basta executar o comando a seguir em uma máquina com Python 3 instalado, na pasta do
projeto:

\begin{center}
  ./compiler.py arquivo\_entrada
\end{center}

Com isso, será gerado, no mesmo diretório, um arquivo chamado \textbf{out.tam}, que contém as instruções de máquina TAM geradas
e que pode, então, ser executado através de um interpretador TAM.

\section{TESTES E RESULTADOS}

Para ilustrar o funcionamento do compilador integrado, são apresentados dois exemplos de compilação.

\subsection{right1.txt}

\begin{itemize}
 \item Código em SmallL:\\
 
  \begin{mdframed}[linecolor=black, leftline=false, rightline=false]
    \inputminted[linenos, fontsize=\footnotesize]{text}{../input/right1.txt}
  \end{mdframed}
  
  \mbox{}
 
 \item Código intermediário: \\
 
 \begin{mdframed}[linecolor=black, leftline=false, rightline=false]
    \inputminted[linenos, fontsize=\footnotesize]{text}{right1.out}
  \end{mdframed}

  \mbox{}
  
  \item Código TAM: \\
 
 \begin{mdframed}[linecolor=black, leftline=false, rightline=false]
    \inputminted[linenos, fontsize=\footnotesize]{text}{right1.tam}
  \end{mdframed}
\end{itemize}

\subsection{right3.txt}

\begin{itemize}
 \item Código em SmallL:\\
 
  \begin{mdframed}[linecolor=black, leftline=false, rightline=false]
    \inputminted[linenos, fontsize=\footnotesize]{text}{../input/right3.txt}
  \end{mdframed}
  
  \mbox{}
 
 \item Código intermediário: \\
 
 \begin{mdframed}[linecolor=black, leftline=false, rightline=false]
    \inputminted[linenos, fontsize=\footnotesize]{text}{right3.out}
  \end{mdframed}

  \mbox{}
  
  \item Código TAM: \\
 
 \begin{mdframed}[linecolor=black, leftline=false, rightline=false]
    \inputminted[linenos, fontsize=\footnotesize]{text}{right3.tam}
  \end{mdframed}
\end{itemize}


\section{CONCLUSÃO}

Nesse trabalho foi possível verificar o funcionamento completo de um compilador. Durante a disciplina, de forma geral,
foram adquiridos conhecimentos fundamentais para a construção de Compiladores e de processadores de linguagens de forma
geral.

\section{REFERÊNCIAS}

\bibliographystyle{sbc}
\bibliography{sbc-template}

\end{document}
